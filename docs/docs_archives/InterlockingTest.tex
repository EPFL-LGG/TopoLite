% XeLaTeX can use any Mac OS X font. See the setromanfont command below.
% Input to XeLaTeX is full Unicode, so Unicode characters can be typed directly into the source.

% The next lines tell TeXShop to typeset with xelatex, and to open and save the source with Unicode encoding.

%!TEX TS-program = xelatex
%!TEX encoding = UTF-8 Unicode

\documentclass{beamer}
\usepackage[utf8]{inputenc}

\title{Interlocking Test}
\author{Ziqi Wang}
%\date{}                                           % Activate to display a given date or no date

\begin{document}
\frame{\titlepage}

% For many users, the previous commands will be enough.
% If you want to directly input Unicode, add an Input Menu or Keyboard to the menu bar 
% using the International Panel in System Preferences.
% Unicode must be typeset using a font containing the appropriate characters.
% Remove the comment signs below for examples.

% \newfontfamily{\A}{Geeza Pro}
% \newfontfamily{\H}[Scale=0.9]{Lucida Grande}
% \newfontfamily{\J}[Scale=0.85]{Osaka}

% Here are some multilingual Unicode fonts: this is Arabic text: {\A السلام عليكم}, this is Hebrew: {\H שלום}, 
% and here's some Japanese: {\J 今日は}.

\begin{frame}
\frametitle{Interlocking Test}
The original interlocking test can be formulated as:\par
\begin{equation*}
\texttt{Interlocking}  \iff  \{x | Ax \geq 0\} = \{\mathbf{0}\} 
\end{equation*}
\end{frame}


\begin{frame}
\frametitle{Linear Programming}
Thanks to Samara, we reformulate the feasibility problem into a linear programming.
\begin{equation}
\begin{aligned}
\min_{t, x}& -\sum_{i=1}^{m} t_i \\
\texttt{s.t.} \hspace{0.5cm} & Ax = t \\
& t \geq 0
\end{aligned}
\label{eq:lp}
\end{equation}\par
The structure is not interlocking \texttt{iff} the optimal value of Eq~\ref{eq:lp} is non zero. \pause \par
\vspace{1cm}\textbf{The challenge here is how to solve this optimization.}
\end{frame}

\begin{frame}
\frametitle{Interior Point Method}
A simple idea would be using the \textbf{interior point method}. However, the interior point method needs a strictly feasible point of Eq~\ref{eq:lp}, which leads to the "chicken and egg" problem. If this strictly feasible point is found in advance, we already can confirm the structure to be non-interlocking without starting the optimization.
\end{frame}

\begin{frame}
\frametitle{Big-M method}
When M tends to be infinity, the solution of right optimization is a strictly feasible point of the left optimization. Meanwhile, the right optimization has a nature strictly feasible point:\\ \centering $\lambda = 1, x = 0, t = t_0$.
\begin{columns}
	\begin{column}{0.5\textwidth}
		\begin{equation*}
		\begin{aligned}
		\min_{t, x}& -\sum_{i=1}^{m} t_i \\
		\texttt{s.t.} \hspace{0.5cm} & Ax = t \\
		& t \geq 0
		\end{aligned}
		\end{equation*}
	\end{column}
	\begin{column}{0.5\textwidth}
		\begin{equation*}
		\begin{aligned}
		\min_{t, x,\lambda}& -\sum_{i=1}^{m} t_i + \lambda M \\
		\texttt{s.t.} \hspace{0.5cm} & Ax + \lambda t_0 = t \\
		& t, \lambda  \geq 0 \\
		& t_0 = (1, \cdots, 1)^T
		\end{aligned}
		\label{eq:lp}
		\end{equation*}
\end{column}
\end{columns}
\end{frame}

\begin{frame}
\frametitle{A New Interlocking Test}
Further, we find a link between interlocking property and the big-M method. If let $t_{\texttt{sol}}(M)$ be the optimal point of the big-M method with respect to a given $M$, we have:
\begin{equation*}
\texttt{Interlocking} \iff \lim_{M\to\infty}t_{\texttt{sol}}(M) = 0
\end{equation*}
\end{frame}

\begin{frame}
\frametitle{Affine Scaling}
For a general linear programming:
\begin{equation*}
\begin{aligned}
\min_{x} & \hspace{0.1cm} c^Tx \\
\texttt{s.t.} \hspace{0.5cm} & Ax = 0 \\
& x \geq 0
\end{aligned}
\end{equation*}
The affine scaling iteration is:
\begin{enumerate}[(I)]
\item<0-> Suppose $x^0$ is a strictly feasible point.
\item<1-> Let ${X_k}$ be the \textbf{diagonal matrix} with $x^k$ on its diagonal.
\item<2-> Let the dual variable $y^k = (AX_k^2A^T)^{-1}AX_k^2c$
\item<3-> The reduce cost $r^k = c - A^Ty^k$
\item<4-> The solution is found if $\mathbf{1}^TX_kr^k < \epsilon$.
\item<5-> Otherwise, Update $x^{k+1} = x^{k} - \beta \frac{X_k^2r^k}{\lVert X_kr^k \lVert}
\end{enumerate}

\end{frame}

\end{document}  